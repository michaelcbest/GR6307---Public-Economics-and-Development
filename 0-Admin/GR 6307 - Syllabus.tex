\documentclass{article}

\usepackage[style=chicago-authordate,maxnames=10]{biblatex}

\addbibresource{readinglist.bib}
\begin{document}

\section*{Course Outline}

This course covers a range of challenges faced by governments in low- and middle-income countries. The course will cover both applied theory papers and empirical papers applying the latest empirical methods.

\section*{Evaluation}

Students will be evaluated on their performance on the class debate on \emph{``A Universal Basic Income is preferable to targeted transfers in middle income countries''}, a referee report, and class participation.

\section{Taxation}
Tax systems in rich countries look very different from those in poor countries. How should tax systems be designed in the presence of high levels of tax evasion and informality? How much tax evasion is there? How can governments reduce tax evasion?

\begin{enumerate}
\item \fullcite{AllinghamSandmo}.
\item \fullcite{Alstadsaeteretal2017}.
\item \fullcite{Andreonietal1998}.
\item \fullcite{Artavanisetal2016}.
\item \fullcite{BenhassineEtal2018}.
\item \fullcite{BesleyPersson2009}.
\item \fullcite{BesleyPersson2011}.
\item \fullcite{BesleyPersson2013}.
\item \fullcite{Bestetal2015}.
\item \fullcite{BoadwaySato2009}.
\item \fullcite{BurgessStern}.
\item \fullcite{Chetty2009}.
\item \fullcite{Dwengeretal2016}.
\item \fullcite{FismanWei2004}.
\item \fullcite{GordonLi2009}.
\item \fullcite{Gorodnichenkoetal2009}.
\item \fullcite{Klevenetal2011}.
\item \fullcite{LaPortaShleifer2014}.
\item \fullcite{LuttmerSinghal2014}.
\item \fullcite{Pomeranz2015}.
\item \fullcite{Sequeira2016}.
\item \fullcite{Slemrod2007}.
\item \fullcite{SlemrodYitzhaki}.
\item \fullcite{Zucman2013}.
\item \fullcite{Zucman2014}.
\end{enumerate}

\section{ Anti-poverty Programs}
Targeted transfers to poor household are a huge part of government spending in low- and middle-income countries. How should these programs be designed? Should they be monetary or in-kind transfers? Should they be means-tested? If so, how will eligibility be determined?

\begin{enumerate}
\item \fullcite{Akerlof1978}
\item \fullcite{Alatasetal2012}
\item \fullcite{Alatasetal2016}
\item \fullcite{AngelucciDeGiorgi2009}
\item \fullcite{AttanasioLechene2014}
\item \fullcite{BairdMcIntoshOzler2011}
\item \fullcite{Banerjeeetal2018}
\item \fullcite{Barnwal2017}
\item \fullcite{Basurtoetal2017}
\item \fullcite{Bertrandetal2013}
\item \fullcite{BesleyCoate1992}
\item \fullcite{Brolloetal2016}
\item \fullcite{Chettyetal2013}
\item \fullcite{Cohenetal2015}
\item \fullcite{Cunhaetal2017}
\item \fullcite{DeshpandeLi2017}
\item \fullcite{Dupasetal2016}
\item \fullcite{GahvariMattos2007}
\item \fullcite{ImbertPapp2015}
\item \fullcite{KlevenKopczuk2011}
\item \fullcite{NicholsZeckhauser1982}
\item \fullcite{ParkerTodd2017}
\item \fullcite{Rothstein2010}
\item \fullcite{Saez2002}
\end{enumerate}

\section{The Personnel Economics of the State}
The government is the largest employer in most countries, but public service delivery is notoriously inefficient. How can governments attract honest, capable and motivated workers? How will the government monitor and incentivize their workers?

\begin{enumerate}
\item \fullcite{AghionTirole1997}
\item \fullcite{Ashrafetal2014}
\item \fullcite{Ashrafetal2016}
\item \fullcite{Banerjeeetal2013}
\item \fullcite{BenabouTirole2006}
\item \fullcite{BesleyGhatak2005}
\item \fullcite{Callenetal2015}
\item \fullcite{Chaudhuryetal2006}
\item \fullcite{DalBoetal2013}
\item \fullcite{DeReeetal2017}
\item \fullcite{Deserrano2017}
\item \fullcite{Dufloetal2013}
\item \fullcite{Dufloetal2016}
\item \fullcite{Dufloetal2012}
\item \fullcite{Finanetal2017}
\item \fullcite{KhanKhwajaOlken2016}
\item \fullcite{KhanKhwajaOlken2017}
\item \fullcite{LazearOyer2012}
\item \fullcite{MuralidharanSundararaman2011}
\item \fullcite{Olken2007}
\item \fullcite{OlkenPande2012}
\item \fullcite{Prendergast2003}
\item \fullcite{Prendergast2007}
\end{enumerate}

\section{Data \& Technology in Government}
Most policy problems involve prediction of a counterfactual (what if we raise tax rates?) or a state of the world (how much poverty is there?). How can machine learning methods help governments make these predictions? Can new technologies be used to monitor government workers and increase their productivity and/or effort?

\begin{enumerate}
\item \fullcite{Abelsonetal2014}
\item \fullcite{Ackermannetal2016}
\item \fullcite{BanerjeeHannaOlkenKyleSumarto2016}
\item \fullcite{Banerjeeetal2017}
\item \fullcite{Blumenstocketal2015}
\item \fullcite{Blumenstock2016}
\item \fullcite{CallenLong2015}
\item \fullcite{Callenetal2016}
\item \fullcite{CasaburiTroiano2016}
\item \fullcite{Chalfinetal2016}
\item \fullcite{Engstrometal2016}
\item \fullcite{Fujiwara2015}
\item \fullcite{Glaeseretal2015}
\item \fullcite{Glaeseretal2016}
\item \fullcite{Goeletal2016}
\item \fullcite{Jeanetal2016}
\item \fullcite{Kangetal2013}
\item \fullcite{Kleinbergetal2015}
\item \fullcite{Kleinbergetal2017}
\item \fullcite{Muralidharanetal2016}
\item \fullcite{Muralidharanetal2017}
\item \fullcite{MuralidharanSinghGanimian2017}
\item \fullcite{Naritomi2016}
\end{enumerate}


%\printbibliography
\end{document}